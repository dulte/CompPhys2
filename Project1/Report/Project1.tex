\documentclass[a4paper, 10pt]{article}
\usepackage[utf8]{inputenc}
\usepackage{verbatim}
\usepackage{listings}
\usepackage{graphicx}
\usepackage{a4wide}
\usepackage{color}
\usepackage{amsmath}
\usepackage{amssymb}
\usepackage[dvips]{epsfig}
\usepackage[toc,page]{appendix}
\usepackage[T1]{fontenc}
\usepackage{cite} % [2,3,4] --> [2--4]
\usepackage{shadow}
\usepackage{hyperref}
\usepackage{titling}
\usepackage{marvosym }
\usepackage{subcaption}
\usepackage[noabbrev]{cleveref}
\usepackage{wasysym}


\renewcommand{\topfraction}{.85}
\renewcommand{\bottomfraction}{.7}
\renewcommand{\textfraction}{.15}
\renewcommand{\floatpagefraction}{.66}
\renewcommand{\dbltopfraction}{.66}
\renewcommand{\dblfloatpagefraction}{.66}
\setcounter{topnumber}{9}
\setcounter{bottomnumber}{9}
\setcounter{totalnumber}{20}
\setcounter{dbltopnumber}{9}


\setlength{\droptitle}{-10em}   % This is your set screw

\setcounter{tocdepth}{2}

\lstset{language=c++}
\lstset{alsolanguage=[90]Fortran}
\lstset{basicstyle=\small}
\lstset{backgroundcolor=\color{white}}
\lstset{frame=single}
\lstset{stringstyle=\ttfamily}
\lstset{keywordstyle=\color{red}\bfseries}
\lstset{commentstyle=\itshape\color{blue}}
\lstset{showspaces=false}
\lstset{showstringspaces=false}
\lstset{showtabs=false}
\lstset{breaklines}
\title{FYS4411 - Project 1\\
Variational Monte Carlo}
\author{Daniel Heinesen, Gunnar Lange}
\begin{document}
\maketitle
\begin{abstract}
Abstract awesomeness
\end{abstract}
\tableofcontents
\section{Introduction}
\section{Theoretical model}
\section{Methods}\label{Method_section}
\section{Results and discussion}
\section{Conclusion}
\subsection{Conclusion}

\subsection{Outlook}

\begin{appendices}
\section{Finding the analytic expression for the local energy}
We wish to find:
\begin{equation}
E_L=\frac{1}{\Psi_T}\nabla^2 \Psi_T
\end{equation}
Where $\Psi_T$ is given by:
\begin{equation}
\exp\left[ -\alpha \sum_{i}^n\left(x_i^2+y_i^2+z_i^2\right)\right]\prod_{i<j}f(a, |\mathbf{r}_i-\mathbf{r}_j|)
\end{equation}
Look first at $a=0$. In this case, this reduces to the harmonic oscillator potential. The local energy is then:
\begin{equation}
\exp\left[ \alpha \sum_{i}^n\left(x_i^2+y_i^2+z_i^2\right)\right]\left(-\frac{\hbar^2}{2m_i}\nabla_i^2+V\right)\exp\left[ -\alpha \sum_{i}^n\left(x_i^2+y_i^2+z_i^2\right)\right]
\end{equation}
In the simple harmonic oscillator case, this gives:
\begin{equation}
E_L(\mathbf{r},\alpha)=-\frac{\hbar^2}{2m}\sum_{i=1}^n 2\alpha \left(2\alpha(x_i^2+y_i^2+z_i^2)-3\right)+\frac{1}{2}m\omega_{ho}^2\sum_{i=1}^n(x_i^2+y_i^2+z_i^2)
\end{equation}
\textbf{FIX FOR ONE/N PARTICLES AND LOWER DIMENSION}
Similarly for the drift force:
\begin{equation}
F=\frac{2\nabla \Psi_T}{\Psi_T}
\end{equation}
\begin{equation}
F=-2\alpha\sum_{i=1}^n(x_i+y_i+z_i)
\end{equation}
Now for the ugly part:\\
For one particle, $\nabla_k$:
\begin{equation}
\nabla_k\Psi_T(\mathbf{r})=\nabla_k \prod_{i=1}^n\phi(\mathbf{r}_i)\exp\left(\sum_{i<j} u(r_{ij})\right)
\end{equation}
Now apply the product rule. For the first term, all terms are unchanged except where $i=k$, giving the first term as:
\begin{equation}
\nabla_k \phi(\mathbf{r}_k)\left[ \prod_{i\neq k} \phi(\mathbf{r}_i)\right]\exp\left(\sum_{i<j}u(r_{ij})\right)
\end{equation}
The second term is trickier. We can rewrite it as follows:
\begin{equation}
\left(\prod_l \phi(\mathbf{r_l})\right)\nabla_k \exp \left(\sum_{i=1}^n \sum_{i<j} u(r_{ij})\right)=\left(\prod_l \phi(\mathbf{r_l})\right) \exp \left(\sum_{i=1}^n \sum_{i<j} u(r_{ij})\right)\nabla_k\left(\sum_{i=1}^n \sum_{i<j} u(r_{ij})\right)
\end{equation} 
To get anything non-zero in the last term, we must have either $i=k$ or $j=k$. If $i=k$, then the term may be rewritten as:
\begin{equation}
\sum_{k<j}\nabla_k u(r_{kj})
\end{equation}
If $j=k$ then all terms such that $i>k$ will vanish, leaving us with:
\begin{equation}
\sum_{i>k} \nabla_k u(r_{ik})
\end{equation}
Putting this all together gives:
\begin{equation}
\nabla_k \left(\sum_{i=1}^n \sum_{i<j} u(r_{ij})\right)=\sum_{j\neq k} \nabla_k u(r_{ij})
\end{equation}
Finally, to get the second derivative:

\end{appendices}
\end{document}
